\chapter{Basic operations}

% ---------------------------------------------------------------------------------------
\section{The very first GET of a ressource}
% ---------------------------------------------------------------------------------------

The simplest possible operation we can do with Ramone is a GET of a resource 
from a well known URL and then decode that resource into a client side object.

As an example we can GET the Twitter timeline for user JornWildt (see the
Twitter specification at 
\url{https://dev.twitter.com/docs/api/1/get/statuses/user\_timeline}):

\lstset{style=sharpc}
\begin{lstlisting}
      // Fixed URL to timeline of JornWildt
      string url = "/1/statuses/user_timeline.json?screen_name=JornWildt";

      // All interaction with Ramone goes through a session
      ISession session
        = RamoneConfiguration.NewSession(new Uri("https://api.twitter.com"));

      // Create a Ramone request by binding variables to a URI template
      Request request = session.Bind(url);

      // GET response from Twitter
      Response response = request.Get();

      // Extract payload as a C# dynamic created from the JSON response
      dynamic timeline = response.Body;

      // Write response to console ...
      Console.WriteLine("This is the timeline for JornWildt:");
      foreach (dynamic tweet in timeline)
        Console.WriteLine("* {0}.", tweet.text);
\end{lstlisting}

Lets go through that code step by step and see what is going on:

\begin{itemize}
	\item The first step creates a session. Sessions are containers for
	things like authorization, base URL and cookies and must always be referenced somehow when
	creating a new request.
	
	\item The next step binds the Twitter URL to the session and creates a request.
	
	\item Then the request is activated by calling one of the HTTP methods - in this
	case GET.
	
	\item Ramone takes care of content negotiation but in this case it is not really
	needed since Twitter always returns application/json.
	
	\item Ramone decodes the application/json using one of a set of predefined codecs.
	In this case it returns a C\# dynamic which reflects the JSON response.
	
	\item At last the returned tweets are printed.
\end{itemize}


% ---------------------------------------------------------------------------------------
\section{Sessions, Services and Binding}
% ---------------------------------------------------------------------------------------

The are a few elements of Web APIs that are orthogonal to the core business domain, 
things such as authentication and cookie handling. These things should be kept out of the 
way when interacting with the API, aiming for clean and terse code that is very
explicit about the Web API business domain.

These "orthogonal elements" are handled by Ramone's Sessions and Services. These two
concepts are very closely related and contains almost the same information; a service 
defines the settings for a specific Web API and a session holds state for a specific
interaction with a service.

\section{Universal Resource Identifiers}

Ramone depends on System.Uri for resource identification and adds no additional semantics to them.


\section{URI Templates}

Ramone depends on System.UriTemplate for URI templates. 
